\chapter*{Project Description}
Asset management across a large geographically distributed organization presents a significant management challenge. In the past, individual divisions have been independently responsible for asset management and tracking. Independent asset tracking will no longer scale for the needs of OSNAP and thus a large scale centralized asset management system is required. The Logistical Operations Service Tracker (LOST) will fill this requirement and enable OSNAP to safely continue to scale out.

OSNAP personnel needing access to LOST will have LOST access added to their WHO account. WHO will provide LOST with user information including username, person name, and division. LOST will internally maintain other user information as needed to support LOST functions.

Many OSNAP assets are classified and visibility into the type and location of those assets must be tightly controlled. LOST will support mandatory access control (MAC) to protect asset information (e.g. asset type and location). Since many assets must be hidden in plane sight (e.g. shipments on public highways), plausible substitute information in some cases will be shown to users with insufficient clearance. The special user role of 'classifier' will be able to change the classification of assets within LOST. The special user role of 'assigner' will be able to change the clearance of users. LOST must enforce that no user is both a classifier and an assigner.

Assets are instances of products. Product information can be used to look up plausible substitutions for an asset within a report. An asset may also include an explicit substitution. Each asset carries a classification, since an otherwise declassified product (e.g. a note pad) may become classified based on how it is used. Products also carry a classification and an asset may not carry a classification that is incompatible with the backing product.

Assets will be associated with a facility while at rest. While in transit, an asset will be associated with a travel request. The travel request will include the starting, ending, and last known location of the asset as well as information regarding the convoy. In addition to the current asset location, the location history of an asset can be reported using LOST.


Existing OSNAP asset data exists in a plethora of division specific data sources. The existing asset data will need to be in LOST at the time each division is cut over. Operators will use data migration scripts to support this activity.